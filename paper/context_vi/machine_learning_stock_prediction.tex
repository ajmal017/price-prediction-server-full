Dự báo giá cổ phiếu là một việc rất phức tạp. Hầu hết các nhà môi giới chứng khoán sử dụng phân tích chuỗi kỹ thuật, cơ bản hoặc phân tích chuỗi thời gian (time series) trong việc cố gắng dự đoán giá cổ phiếu. Tuy nhiên, các chiến lược này không dẫn đến kết quả đáng tin cậy vì chúng hướng dẫn về xu hướng và không phải là giá có độ chính xác cao nhất. Cần phải sử dụng các phương pháp nâng cao để dự đoán kết quả chính xác nhất. Các nhà nghiên cứu đã sử dụng các phương pháp khác nhau và các bộ tham số đầu vào khác nhau để dự đoán giá cổ phiếu trong vài thập kỷ qua.

Adebiyi và Adewumi [3] đã xây dựng mô hình dự đoán giá cổ phiếu bằng cách sử dụng mô hình Autoregressive integrated moving average (ARIMA). Kết quả cho thấy mô hình ARIMA có tiềm năng mạnh mẽ để dự đoán ngắn hạn.

Xing, Sun, Wang và Yu. [24] giới thiệu một loại phương pháp dựa trên Mô hình Markov ẩn để dự báo xu hướng giá cổ phiếu. Khác với mô hình dự đoán cổ phiếu hiện tại, họ đã cố gắng tìm mối quan hệ ẩn tồn tại giữa các giá cổ phiếu bằng Mô hình Markov ẩn. Kết quả thử nghiệm cho thấy, phương pháp này có thể nhận được kết quả khá chính xác, đặc biệt hiệu quả trong dự đoán thời gian ngắn.

Marček [4] đã mô tả khái niệm cơ bản của mô hình hồi quy tuyến tính mờ dựa trên nguyên tắc mở rộng tham số mờ. Bài viết đã trình bày mô hình tự phát (AR) sử dụng nguyên tắc mở rộng tham số mờ và Fuzzy Neural Network (FNN) để ước tính và dự đoán giá cổ phiếu. Qua đó cho thấy rằng kết quả ban đầu từ việc sử dụng kiến trúc FNN tốt hơn kiến trúc Artificial Neural Network (ANN) căn bản cho việc dự đoán hằng ngày.

Hegazy, Soliman và Salam [5] đã triển khai mô hình học máy để dự đoán giá thị trường chứng khoán. Bằng cách kết hợp thuật toán Particle swarm optimization(PSO) và thuật toán Least square support vector machine (LS-SVM). Thuật toán PSO chọn ba tham số bất kì kết hợp tốt nhất từ nghiên cứu về dữ liệu lịch sử của cổ phiếu và các chỉ số kỹ thuật cho LS-SVM để tránh các vấn đề over-fitting. Mô hình được đề xuất đã được áp dụng và đánh giá bằng cách sử dụng mười ba bộ dữ liệu tài chính và so sánh với mạng nơ ron nhân tạo với thuật toán Levenberg - Marquest (LM). Kết quả thu được cho thấy mô hình đề xuất có độ chính xác dự đoán tốt hơn và tiềm năng của thuật toán PSO trong việc tối ưu hóa LS-SVM.

Tiwari, Bharadwaj và Gupta [22] đề xuất mục đích của việc phân tích dữ liệu được sử dụng để hỗ trợ các nhà đầu tư đưa ra dự đoán tài chính chính xác để đưa ra các quyết định đầu tư đúng đắn. Hai nền tảng đã được sử dụng: Python và R

Các kỹ thuật khác nhau như Arima, Holt winters, mạng nơ-ron (Feed forward và Multi-layer perceptron), hồi quy tuyến tính và time series được triển khai để dự báo hiệu suất giá chỉ số mở trong R. Mặc dù trong Python Multi-layer perceptron và hồi quy vectơ đã được triển khai để dự báo giá cổ phiếu Nifty 50 và phân tích cá nhân của cổ phiếu đã được thực hiện bằng cách sử dụng các tweet gần nhất trên Twitter. Các chỉ số chứng khoán Nifty 50 (ANSEI) được coi là đầu vào dữ liệu cho các phương thức được triển khai. Chín năm dữ liệu đã được sử dụng. Độ chính xác được tính bằng cách sử dụng 2-3 năm kết quả dự báo của R và 2 tháng kết quả dự báo của Python sau khi so sánh với giá thực tế của các cổ phiếu.

Mittal và Goel [6] đã áp dụng các nguyên tắc phân tích cá nhân và machine learning để tìm ra mối tương quan giữa ‘public sentiment’ and ‘market sentiment’. Dữ liệu Twitter được sử dụng để dự đoán tâm trạng công cộng và những ngày trước đó Giá trị trung bình công nghiệp (DJIA) của Dow Dow Jones đã được sử dụng để dự đoán diễn biến của thị trường chứng khoán. Họ đề xuất một phương pháp xác thực chéo mới cho dữ liệu tài chính để kiểm tra kết quả và thu được độ chính xác 75,56\% bằng cách sử dụng Self Organizing Fuzzy Neural Networks (SOFNN) trên các nguồn cấp dữ liệu Twitter và DJIA từ giai đoạn tháng 6 năm 2009 đến tháng 12 năm 2009.

Wamkaya và Lawrence [7] đã đề xuất việc sử dụng Mạng nơ ron nhân tạo là feedforward multi-layer perceptron with error backpropagation (Ở trên thì nó viết là feedforward and multi-layer perceptron). Họ đã phát triển một mô hình cấu hình 5: 21: 21: 1 với 80\% dữ liệu đào tạo trong 130.000 chu kỳ. Nghiên cứu đã phát triển một nguyên mẫu và thử nghiệm nó trên dữ liệu 2008 - 2012 từ các thị trường chứng khoán nơi kết quả dự đoán cho thấy MAPE trong khoảng 0,71\% đến 2,77\%.

Dunne [27] đã phân tích các phương pháp dự đoán thị trường chứng khoán hiện có và các phương pháp mới. Ông đã thực hiện ba cách tiếp cận khác nhau : Phân tích cơ bản, Phân tích kỹ thuật và ứng dụng Machine Learning. Ông tìm thấy bằng chứng ủng hộ mô hình yếu của Efficient Market Hypothesis, rằng giá lịch sử không chứa thông tin hữu ích nhưng ngoài dữ liệu mẫu có thể là dự đoán. Ông đã chỉ ra rằng Fundamental Analysis và Machine Learning có thể được sử dụng để hướng dẫn các nhà đầu tư đưa ra quyết định. Ông chứng minh một lỗ hổng phổ biến trong phương pháp Phân tích Kỹ thuật và cho thấy rằng nó tạo ra thông tin hữu ích nhưng hạn chế. Dựa trên những phát hiện của ông, các chương trình giao dịch thuật toán đã được phát triển và mô phỏng bằng Quantopian.

Selvin, Vinayakumar, Gopalakrishnan, Menonans và Soman K.P [8] đã thử nghiệm một mô hình tiếp cận độc lập. Thay vì lắp dữ liệu vào một mô hình cụ thể, họ đã xác định động lực tiềm ẩn hiện có trong dữ liệu bằng kiến trúc deep learning. Họ đã sử dụng ba kiến trúc deep learning khác nhau để dự đoán giá của các công ty niêm yết NSE và so sánh hiệu suất của họ. Ba mô hình là NN, LSTM và CNN. Hiệu suất của các mô hình đã được định lượng bằng lỗi phần trăm. Giá trị tối đa của tỷ lệ phần trăm lỗi đã thu được cho mỗi mô hình. Kết quả cho thấy CNN đã cho kết quả chính xác hơn so với hai mô hình còn lại.

Kita, Zuo, Harada và Mizuno [9] đã phát triển thuật toán dự đoán giá cổ phiếu bằng cách sử dụng mạng Bayes. Thuật toán sử dụng mạng hai lần. Đầu tiên, mạng được xác định từ giá cổ phiếu hàng ngày và sau đó nó được áp dụng để dự đoán giá cổ phiếu hàng ngày đã được quan sát. Các lỗi dự đoán được đánh giá từ giá cổ phiếu hàng ngày và dự đoán của nó. Thứ hai, mạng được xác định một lần nữa từ cả giá cổ phiếu hàng ngày và lỗi dự đoán hàng ngày và sau đó nó được áp dụng cho dự đoán giá cổ phiếu trong tương lai. Kết quả bằng số cho thấy lỗi dự đoán tối đa của thuật toán hiện tại là 30\%.


Xiongwen Pang, Yanqiang Zhou, Pan Wang, Weiwei Lin và Victor Chang đã đưa ra mô hình ELSTM bằng làm giảm chiều dữ liệu của chỉ số tỗng hợp và các mã cổ phiếu bằng word vector, rồi đưa ra dự đoán bằng LSTM. Kết quả đạt được chỉ số MSE = 0.017, tốt hơn so với LSTM truyền thống
