Thị trường chứng khoán là các tổ chức giao dịch nơi chứng khoán (vốn chủ sở hữu) và tài chính khác
các công cụ như trái phiếu được cung cấp cho thương mại. Đối với cổ phiếu, thị trường thường hoạt động
một giao dịch người mua đưa ra mong muốn mức giá muốn mua, người bán đưa ra mức giá muốn bán, và nếu có người mua và người bán đều có mức giá phù hợp với nhau thì giao dịch sẽ được diễn ra. Nếu không thì sẽ không có giao dịch nào diễn ra và chờ đợi một mức giá  trong tương lai hoặc hết hạn.

Trong hầu hết các sàn giao dịch chứng khoán, thị trường phổ biến và dễ tiếp cận là thị trường chứng khoán
(cổ phiếu), có rất ít rào cảng để mọi người có thể tham gia. Thị trường chứng khoán là
do đó tích cực hơn, có nhiều người chơi và do đó một phân khúc xứng đáng để nghiên cứu thêm.
Hiệu suất của thị trường chứng khoán được đo lường hàng ngày bởi một số chỉ số chính
chẳng hạn như 'chỉ số tổng hợp', chỉ số thị trường chứng khoán của tất cả các cổ phiếu được giao dịch tại Sở giao dịch chứng khoán. Một chỉ số như vậy rất quan trọng trong việc không chỉ đo lường
hiệu suất của các giao dịch trên thị trường chứng khoán mà còn phản ánh rõ các hoạt động kinh tế của một quốc gia và các mối quan hệ quốc tế. Cổ đông tuy nhiên không trực tiếp thực hiện giao dịch,
cũng không có cuộc họp nào giữa người mua và người bán để đàm phán. Cổ đông giao dịch
bằng cách đưa ra hướng dẫn cho các Môi giới chứng khoán của họ, những người lần lượt thực hiện các lệnh. Môi giới chứng khoán
thường cũng tư vấn cho khách hàng về nơi giao dịch. Trong vai trò tư vấn của họ, một số môi giới chứng khoán
căn cứ lời khuyên của họ về các nguyên tắc cơ bản của các cổ phiếu khác nhau hoặc thực hiện kỹ thuật
phân tích. Tuy nhiên, không có phương pháp dự đoán nào trong số này đảm bảo lợi nhuận vì chúng
thường chỉ cho thấy một xu hướng trong tương lai và khả năng tăng hoặc giảm giá chứ không phải là
giá cổ phiếu dự kiến ​​trong tương lai. Môi giới chứng khoán cần được trao quyền, thông qua tốt hơn
công cụ dự đoán, để cho phép họ có một số khả năng để cung cấp lời khuyên tốt nhất cho họ
khách hàng.Mọi người người môi giới chứng khoán đều mong muốn có một công cụ dự đoán mà có thể sử dụng để phán đoán về biến động giá chính xác như là một cơ sở của đầu tư. 
Đây là công cụ mọi người đều mong muốn có kể từ khi thị trường chứng khoán ra đời. Các nhà toán học từ những năm 1960 đã đưa ra các mô hình thống kê như ARIMA, SARIMA để nắm bắt thị trường nhưng kết quả đạt được lại không như mong đợi.
Kể từ năm 2010, máy học đã đạt đươc những đôt phá nhất định. Ý tưởng chinh phục cổ phiếu đã quay trở lại và mãnh liệt hơn khi học sâu đã vượt qua hàng loạt các thuật toán khác chứng tỏ sự yêu việc của mình. Đặc biệt khi sử dụng các mạng thần kinh tuần tự cho các vấn đề chuỗi thời gian cho kết quả chính xác ngoài mong đợi
Nhưng việc chọn mô hình và đưa ra các siêu tham số là một công viêc cần có kinh nghiệm chuyên sâu về cả cổ phiếu lẫn máy học. 
Không phải ai cũng có khả năng đó. Vì vậy chúng tối đễ xuất tối sử dụng tối ưu hoá Bayesian và tìm cách xây dựng mô hình sử dụng LSTM dễ tạo các mô hình có nhiều siêu tham số để tối ưu để mô hình đưa ra
